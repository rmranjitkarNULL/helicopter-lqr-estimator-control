\documentclass[12pt]{article}
\usepackage{amsmath, amssymb, graphicx, float, geometry, setspace}
\geometry{margin=1in}
\setstretch{1.15}

% ---------------------------------------------
%               TITLE PAGE
% ---------------------------------------------
\begin{document}
\begin{titlepage}
    \centering
    {\large Worcester Polytechnic Institute}\\[0.25cm]
    {\large Department of Electrical and Computer Engineering}\\[1.5cm]

    {\Large \textbf{ECE 505 – Analysis of Deterministic Signals and Systems}}\\[0.5cm]
    {\Large \textbf{Final Project Report}}\\[1cm]

    {\Large \textbf{Helicopter Control System:\\
    Stability Analysis, LQR Control, Estimation,\\
    Output Feedback, and Reference Tracking}}\\[1.5cm]

    {\large Instructor: Dr.\ Bo Tang}\\[0.5cm]

    {\large Student: Ryan Ranjitkar}\\[2cm]

    {\large \today}

    \vfill
\end{titlepage}

\tableofcontents
\newpage

% ---------------------------------------------
%               INTRODUCTION
% ---------------------------------------------
\section{Introduction}
This report presents a complete analysis and control design for a multivariable helicopter system
using state-space methods covered in ECE~505. The system represents a twin-engine helicopter
with eight states, four control inputs, and six outputs. The objective is to evaluate stability,
determine controllability and observability, design a stabilizing Linear Quadratic Regulator (LQR),
compute a full-state estimator, synthesize a dynamic output-feedback controller, and finally design
a reference-tracking controller for the first output channel.

The helicopter considered in this study is a high-dimensional, multi-input multi-output (MIMO)
linearized model representative of a twin-engine aircraft equipped with independent rotor blade
controls. Such systems are inherently complex due to aerodynamic coupling, nonlinearities, and
cross-axis interactions. The purpose of this project is to explore how classical state-space techniques
can be used to stabilize and control such a system, which would otherwise be extremely difficult for
manual control.

The MATLAB implementation follows a structured workflow, beginning with open-loop
characterization and concluding with a fully integrated observer-based controller capable of
accurate reference tracking. Each step illustrates the practical considerations behind real-world
flight control systems.

% ---------------------------------------------
%           SYSTEM DESCRIPTION
% ---------------------------------------------
\section{System Description}

\subsection{State-Space Model}
The helicopter dynamics are given by:
\[
\dot{x} = Ax + Bu, \qquad y = Cx + Du
\]

The system has 8 states, 4 inputs, and 6 outputs.  
The state variables consist of angular attitudes, angular rates, and translational velocities,
capturing rigid-body rotorcraft dynamics. The system's structure shows strong coupling between
attitude, velocity, and rotor-induced aerodynamic forces, making decoupled single-input control
infeasible.

The matrices below represent the full linearized dynamics.

\subsubsection*{Matrix \(A\)}
{\small
\[
A =
\begin{bmatrix}
0 & 0 & 0 & 0.9986 & 0.05338 & 0 & 0 & 0 \\
0 & 0 & 1 & -0.003182 & 0.05952 & 0 & 0 & 0 \\
0 & 0 & -11.5705 & -2.5446 & -0.06360 & 0.10678 & -0.09492 & 0.007108 \\
0 & 0 & 0.43936 & -1.99818 & 0 & 0.016651 & 0.018462 & -0.001187 \\
0 & 0 & -2.0409 & -0.458999 & -0.73503 & 0.019256 & -0.004596 & 0.002120 \\
-32.1036 & 0 & -0.503355 & 2.29786 & 0 & -0.021216 & -0.021168 & 0.015812 \\
0.102161 & 32.05783 & -2.34722 & -0.503611 & 0.834948 & 0.021227 & -0.037880 & 0.000354 \\
-1.91097 & 1.713829 & -0.004005 & -0.057411 & 0 & 0.0139896 & -0.0009068 & -0.290514
\end{bmatrix}
\]
}

\subsubsection*{Matrix \(B\)}
{\small
\[
B =
\begin{bmatrix}
0 & 0 & 0 & 0 \\
0 & 0 & 0 & 0 \\
0.124335 & 0.082786 & -2.75248 & -0.017889 \\
-0.036359 & 0.475095 & 0.014291 & 0 \\
0.304492 & 0.014958 & -0.496518 & -0.206742 \\
0.287735 & -0.544506 & -0.016379 & 0 \\
-0.019073 & 0.016367 & -0.544536 & 0.234842 \\
-4.82063 & -0.000381 & 0 & 0
\end{bmatrix}
\]
}

\subsubsection*{Matrix \(C\)}
{\small
\[
C =
\begin{bmatrix}
0 & 0 & 0 & 0 & 0 & 0.0595 & 0.05329 & -0.9968 \\
1 & 0 & 0 & 0 & 0 & 0 & 0 & 0 \\
0 & 1 & 0 & 0 & 0 & 0 & 0 & 0 \\
0 & 0 & 0 & -0.05348 & 1 & 0 & 0 & 0 \\
0 & 0 & 1 & 0 & 0 & 0 & 0 & 0 \\
0 & 0 & 0 & 1 & 0 & 0 & 0 & 0
\end{bmatrix}
\]
}

\subsubsection*{Matrix \(D\)}
\[
D = 0_{6 \times 4}
\]

% ---------------------------------------------
%           STABILITY ANALYSIS
% ---------------------------------------------
\section{Stability Analysis}

\subsection{Eigenvalues of \(A\)}
The eigenvalues of the open-loop system are:
\[
\lambda(A) =
\begin{cases}
-11.4968,\; -2.3036,\; -0.7104,\; -0.2923, \\
-0.1593 \pm 0.5990i,\; 0.2342 \pm 0.5513i
\end{cases}
\]

Two eigenvalues have positive real parts, demonstrating that the system is inherently unstable.
Several others lie close to the imaginary axis, revealing lightly damped oscillatory modes that
would make the helicopter difficult to control without feedback. This behavior is consistent with
real aircraft, which often rely on active stabilization systems.

\subsection{BIBO Stability}
MATLAB returned:
\[
\texttt{System is BIBO UNSTABLE}
\]
because unstable internal poles guarantee that bounded inputs can produce unbounded outputs.

% ---------------------------------------------
%           TRANSFER FUNCTION
% ---------------------------------------------
\section{Transfer Function \(G_{11}(s)\)}

\[
G_{11}(s) =
\frac{4.821 s^7 + 69.23 s^6 + 167.5 s^5 + 106.2 s^4 + 86.83 s^3 + 52.51 s^2 + 18.26 s + 12.6}
{s^8 + 14.65 s^7 + 38.91 s^6 + 32.07 s^5 + 24.32 s^4 + 16.02 s^3 + 6.919 s^2 + 3.694 s + 0.7579}
\]

This transfer function reveals the high-order nature of the helicopter’s internal dynamics. The
multiple poles represent various aerodynamic and inertial modes, including oscillatory
components, and emphasize why simple PID control would be insufficient.

% ---------------------------------------------
%           STEP RESPONSE
% ---------------------------------------------
\section{Open-Loop Step Response}
\begin{figure}[H]
\centering
\includegraphics[width=0.7\textwidth]{Part 5: Step Response (All Inputs).png}
\caption{Open-loop output responses to a unit step input.}
\end{figure}

The open-loop step response exhibits divergence and oscillation, which is expected due to the
unstable poles identified earlier. This reinforces the need for active stabilization.

% ---------------------------------------------
% CONTROLLABILITY / OBSERVABILITY
% ---------------------------------------------
\section{Controllability and Observability}

\[
\texttt{(A,B) CONTROLLABLE}, \qquad 
\texttt{(A,C) OBSERVABLE}
\]

Full controllability ensures that every state can be influenced by the input vector, enabling pole
placement and LQR control. Full observability confirms that all states can be reconstructed from
measurements, allowing reliable estimator design.

% ---------------------------------------------
%               LQR CONTROLLER
% ---------------------------------------------
\section{LQR State-Feedback Control}

The LQR controller computes the optimal feedback matrix \(K\) to minimize a quadratic cost
balancing state deviation and control effort. Choosing \(Q = C^T C\) emphasizes regulation of
measured output variables, while \(R = 0.1I\) penalizes excessive actuator use without overly
restricting control authority.

\subsection{Gain Matrix \(K\)}
\[
K =
\begin{bmatrix}
0.0237 & 0.0313 & -0.0003 & 0.0294 & 0.1442 & 0.1851 & 0.1684 & -3.0898 \\
4.5931 & -0.3917 & 0.0245 & 2.6117 & 0.0809 & -0.0160 & 0.0333 & 0.0017 \\
-0.3604 & -4.5365 & -1.2938 & 0.0189 & -0.4927 & -0.0328 & -0.0243 & -0.0323 \\
-0.1894 & -0.2906 & 0.1756 & -0.0082 & -1.2052 & 0.0034 & -0.0079 & -0.0733
\end{bmatrix}
\]

\subsection{Closed-Loop Eigenvalues}
\[
\lambda(A-BK) = 
\begin{cases}
-15.4005,\; -14.3686,\; -2.6529,\; -0.9874, \\
-0.3928 \pm 0.4660i,\; -0.3816 \pm 0.4078i
\end{cases}
\]

These eigenvalues show strong damping and negative real parts, indicating that the LQR controller
achieves rapid, stable convergence.

\begin{figure}[H]
\centering
\includegraphics[width=0.7\textwidth]{Part 7: Closed-Loop Step Response (LQR).png}
\caption{Closed-loop LQR step response.}
\end{figure}

% ---------------------------------------------
%               OBSERVER
% ---------------------------------------------
\section{Full-State Estimator}

Because not all states are measurable, a full-state estimator (Luenberger observer) is constructed.
The estimator poles were selected as three times the magnitude of the LQR closed-loop poles,
ensuring the observer dynamics are significantly faster than the controlled system’s dynamics.

\subsection{Observer Gain \(L\)}
{\small
\[
L =
\begin{bmatrix}
0.0218 & 25.1799 & -17.5708 & 0.0065 & 0.0067 & 1.0062 \\
-0.0224 & -17.5708 & 25.8865 & 0.1077 & 0.9931 & -0.0049 \\
2.2428 & 0.0067 & -0.0069 & -5.0673 & -8.5456 & -2.1556 \\
2.2441 & 0.0073 & -0.0075 & -5.5362 & 1.3133 & 1.0036 \\
-16.0347 & -0.0469 & 0.0481 & 36.4455 & -6.9629 & -4.0721 \\
25.5205 & -32.1023 & -0.0013 & 21.6036 & 18.4949 & 56.6423 \\
31.1588 & 0.1033 & 32.0566 & 22.1661 & -14.2314 & 62.7225 \\
-6.7984 & -1.9327 & 1.7361 & 18.4984 & -1.8654 & 5.1659
\end{bmatrix}
\]
}

\subsection{Augmented Closed-Loop Eigenvalues}
\[
\lambda(A_{\text{aug}}) = 
\{-46.20,\;-43.10,\;-15.40,\;-14.37,\;-7.96,\;-2.96,\;-2.65,\ldots\}
\]

All eigenvalues lie safely in the left half-plane, confirming stable observer–controller integration.

% ---------------------------------------------
%           OUTPUT FEEDBACK RESPONSE
% ---------------------------------------------
\section{Output Feedback Response}

\begin{figure}[H]
\centering
\includegraphics[width=0.7\textwidth]{Part 8: Output Feedback Response.png}
\caption{Closed-loop output-feedback response.}
\end{figure}

The observer-based controller successfully replicates the LQR closed-loop behavior. The estimator
rapidly reconstructs the system states, enabling effective stabilization using only output
measurements.

% ---------------------------------------------
%           ZERO-INPUT RESPONSE
% ---------------------------------------------
\section{Zero-Input Response}

\begin{figure}[H]
\centering
\includegraphics[width=0.7\textwidth]{Part 10: Zero-Input Response.png}
\caption{Zero-input response of the output-feedback closed-loop system.}
\end{figure}

The system quickly returns to equilibrium after being perturbed, demonstrating strong
disturbance-rejection characteristics.

% ---------------------------------------------
%           REFERENCE TRACKING
% ---------------------------------------------
\section{Reference Tracking for \(y_1(t)\)}

The reference gain is:
\[
N = 3.1651
\]

This gain ensures that the closed-loop system accurately tracks a step reference applied to the first
output channel. The controller modifies the effective DC gain of the closed-loop system so that the
steady-state error becomes zero.

\begin{figure}[H]
\centering
\includegraphics[width=0.7\textwidth]{Part 11: Reference Tracking (y1).png}
\caption{Tracking response of \(y_1(t)\) with step reference.}
\end{figure}

% ---------------------------------------------
%               CONCLUSION
% ---------------------------------------------
\section{Conclusion}
The helicopter system was shown to be open-loop unstable but fully controllable and observable.
An LQR controller successfully stabilized the dynamics, while a full-state estimator enabled
implementation of a dynamic output-feedback controller suitable for real-world applications where
not all states are measurable.

The combined controller–estimator structure achieved fast settling times, strong damping, and
robust asymptotic stability. Reference tracking for the first output was accomplished through a
feedforward gain that eliminates steady-state error. The complete workflow demonstrates how
state-space tools can transform an unstable aircraft model into a well-regulated system capable of
precise command tracking.

\end{document}
